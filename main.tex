\documentclass[format=acmsmall, review=false, screen=true]{acmart}

\usepackage{booktabs} % For formal tables

\usepackage[utf8]{inputenc}

% Metadata Information
\copyrightyear{2018}
%\acmArticleSeq{9}

% Copyright
%\setcopyright{acmcopyright}
\setcopyright{acmlicensed}
%\setcopyright{rightsretained}
%\setcopyright{usgov}
%\setcopyright{usgovmixed}
%\setcopyright{cagov}
%\setcopyright{cagovmixed}

% Paper history
\received{August 2018}

\newcommand \Elisp {Elisp}

% Document starts
\begin{document}
% Title portion. Note the short title for running heads
\title{Evolution of Emacs Lisp}

\author{Stefan Monnier}
\affiliation{%
  \institution{Université de Montréal}
  \streetaddress{C.P.\ 6128, succ.\ centre-ville}
  \city{Montréal}
  \state{QC}
  \postcode{H3C 3J7}
  \country{Canada}}
\email{monnier@iro.umontreal.ca}
\author{Michael Sperber}
\affiliation{%
  \institution{Active Group GmbH}
  \streetaddress{Hechinger Str.\ 12/1}
  \city{Tübingen}
  \country{Germany}
}
\email{sperber@deinprogramm.de}


\input abstract

\ccsdesc{Social and professional topics}
\ccsdesc{Professional topics}
\ccsdesc{History of computing}
\ccsdesc{History of programming languages}

%
% End generated code
%


\keywords{history of programming languages, Lisp, Emacs Lisp}


\maketitle

\section{Introduction}

\section{Prehistory}

%% Mocklisp, Maclisp, Scheme, TECO's language?

\section{Early history}         % -1992 ?

%% * Language & Implementation Overview
%% ** ... usual stuff ...
%% ** Buffer-local variables

%% ** Comparison to other Lisps of the time

%% ** Language implementation
%% ** Interpreter
%% ** Image dumping
%% ** Byte-code architecture (or should that go XEmacs-period?)

\section{XEmacs period}         % 1992-2007 ?

%% How did XEmacs bootstrap?
%% Strings with text-properties?

\section{post-XEmacs}           % 2007-now ?

%% FIXME: I'm putting chunks of text here without knowing where they
%% should really go.

Emacs being the brain child of Richard Stallman, its design strives to
embody and showcase the ideals of Free Software.  For example, not only
should it be legal to get and modify the source code, but every effort
should be made to encourage the end-user to do so.  This has a profound
influence on the \Elisp{} language:
\begin{itemize}
\item The language should be accessible to a wide audience, so that as many
  people as possible can adapt Emacs to their own needs, without being
  dependent on the availability of someone with a technical expertise.
  This can be seen concretely in the inclusion in Emacs of the
  \emph{Introduction to Programming in Emacs Lisp}
  tutorial~\citep{ElispIntro} targeting users with no programming
  experience.  This has been a strong motivation to keep \Elisp{} on the
  minimalist side and to resist incorporation of many Common-Lisp features.
\item It should be easy for the end-user to find the relevant code in order
  to modify Emacs's behavior.   This has driven the development of element
  such as the \emph{docstrings} and more generally the self-documenting
  aspect of the language.  It also imposes constraints on the evolution of
  the language: the use of some facilities, such as \emph{advice}, is
  discouraged because it makes the code more opaque.
\item Emacs should be easily portable to as many platforms as possible.
  This largely explains why \Elisp{} is still using a fairly naive
  mark\&sweep garbage collector, and why its main execution engine is
  a simple bytecode interpreter.
\end{itemize}

%% How to structure that?

%% ** Other implementations
%% *** Elisp in MIT Scheme
%%  Edwin, there's also a paper on this:
%%  https://archive.org/stream/bitsavers_mitaiaimAI_794650/AITR-1451_djvu.txt
%% In "down with emacs lisp" you also mention JEmacs.
%% *** Elisp in Guile
%% *** Elisp in Common-Lisp (Sam Steingold?)

%% * Language Evolution
%% ** CL
%% ** EIEIO/CLOS
%% ** Static scope
%% Mention Neubauer/Sperber ICFP 2011 paper
%% ** How 'bout evolution of typical programming style?
%% ** frame-local variables?
%% ** Evolution of the "core Elisp" language?
%% I'm thinking here of how when/unless/dolist/push/setf slowly migrated from
%% CL to subr.el in Emacs.
%% ** Of course, I'd mention `pcase` in there as well
%% ** lack of tail-call elimination?
%% ** lack of modules?
%% ** what about tooling?
%% *** docstrings (and checkdoc)
%% *** Edebug
%% *** Advice?
%% *** the various `declare` thingies
%% indent, debug, doc-string, advertized-calling-convention, ...


\subsection{Implementation}



%% ** Bootstrap?
%% ** C FFI
%% ** JIT compiler attempts
%% ** Byte code changes
%% ** GC changes

\section{Conclusion}

%% * Future evolution
%% ** Multi-threading?
%% ** OCaml extensions?
%% ** Replacement by Scheme/Guile
%% ** Replacement by Common Lisp

\end{document}
